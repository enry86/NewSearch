\documentclass{acm_proc_article-sp-sigmod07}

\usepackage{listings}
\lstset{
language=XML,
basicstyle=\footnotesize,
numbers=left,
numberstyle=\footnotesize,
breaklines=true,
breakatwhitespace=false
}

\begin{document}

\title{Research Project in Data, Media and Knowledge}
\numberofauthors{1}
\author{Enrico Sartori}

\maketitle

\begin{abstract}
This article presents an implementation of a search engine for news. The
novelty in this approach is that the indexing of the documents is
performed considering the concepts in the analyzed content.
\end{abstract}

\section{PROJECT STRUCTURE}
In order to implement a complete search engine is necessary to realize a
set of different cooperating tools.
First of all a software able to retrieve the articles is needed, at this
scope an rss crawler has been implemented.
In order to speed up the crawling phase, the software uses asynchronous
sockets and multithreading.

When the initial corpus is available is necessary to extract the entities
contained in the various documents, using the services already available
on the web. The integration of these services into this platform has been
reached using the SOAP interfaces available.

Entities and documents are the basic data used for the construction of the
hierarchy of concepts, which is the core of the retrieval system.
The main section of the project is, in fact devoted to the construction of
such a structure, exploiting the existing relationships between entities
and concepts.

The last part of the project regards the handling and the interpretation
of the queries expressed by the user. The query should be represented in
term of concepts and relationships between each other.

\section{INPUT DATA}
The input data, is composed by the text of the document itself, and set of
entities appearing into it.
The process of retrieving the entities can be performed using different
existing services developed by the semantic web community.
In particular has been used the services offered by OpenCalais and Okkam.
Both these softwares offer a SOAP web service, which eases the
programmatic access to these resources.

\subsection{OpenCalais}
The main method exported by this platform is the following one:
\begin{verbatim}
Enlighten(key, content, configuration)
\end{verbatim}
The most important argument of this function is content, which is the
actual content of the article analyzed. The value returned by this call is
a file containing the entities inside the text.

The service can handle different output formats like XML, JSON, or
Microformats. An XML representation of an entity can look like the
following:
\begin{lstlisting}
<rdf:Description
rdf:about="http://d.opencalais.com/dochash-1/87df7f5d-b838-3204-9d60-a2dcacb11257/Instance/5">
<rdf:type rdf:resource="http://s.opencalais.com/1/type/sys/InstanceInfo"/>
<c:docId
rdf:resource="http://d.opencalais.com/dochash-1/87df7f5d-b838-3204-9d60-a2dcacb11257"/>
<c:subject
rdf:resource="http://d.opencalais.com/pershash-1/29d9574d-16e3-3737-9927-2d51b07be0e6"/>
<!--Person: Dmitry Medvedev; -->
<c:detection>
[U.S. President Barack Obama, Russian President ]Dmitry Medvedev[ and
Kazakh President Nursultan Nazarbayev]
</c:detection>
<c:prefix>U.S. President Barack Obama, Russian President </c:prefix>
<c:exact>Dmitry Medvedev</c:exact>
<c:suffix> and Kazakh President Nursultan Nazarbayev</c:suffix>
<c:offset>2020</c:offset>
<c:length>15</c:length>
</rdf:Description>
\end{lstlisting}

In this example the entity related to Medvedev, the field c:subject gives
a opencalais unique id to this entity. Then are given some information
about the localization of the referred token in the text.

In particular the service isolates the previous and following sentences,
which can be useful to find occurrences of keywords related to this
entity.

\subsection{Okkam and Cogito}

\subsection{Formal representation}
The input data can be represented formally as a tuple of this kind:
$$
<D, E, f>
$$
Where $D$ is the initial document, $E$ is the set of entities contained in
the text.
$f$ is function defined as follows:
$$
f: E \rightarrow N
$$

\section{CONCEPTS}
The system presented grounds on the idea of concepts, seen as an
aggregation of entities linked together in term of their meaning.
This approach leads to a flexible hierarchy which is used to index and
retrieve documents.

A formal representation of a concept is the following tuple:
$$
<E, C, g, h>
$$
$E$ is the set of entities related to this Concept, in higher levels of
the hierarchy many concepts will have no relationships with single
entities.

$C$, instead, is the set of concepts which share a relationship with the
current one. The relationship between concept is the basis of the
hierarchical structure aforementioned.

$g$ is a function defined as follows:
$$
g: E \rightarrow \mathcal{R}
$$
It returns the weight of the relationship between an entity and the
current concept.

The value of the relationship between the concepts and the ones appearing
in the set $C$ is given by the function $h$. The formal definition of this
function is:
$$
h: C \rightarrow \mathcal{R}
$$

\section{QUERIES}
The user can express his queries in the form of relationship between
concepts, and the intensity of these relations. 
The query composed by the user can be therefore seen as a graph indicating
the concepts that are required to be related to the documents, and the
distance between each other.

Formally speaking a query can be seen as a tuple in the following form:
$$
<C, f>
$$
Where $C$ is the set of selected concepts, and the function $f$ gives the
distance between each of the concepts.
$$
f: C \times C \rightarrow \mathcal{R}
$$

ciao


\bibliographystyle{abbrv}
\bibliography{report}

\newpage
\null
\newpage


\end{document}
