\documentclass{beamer}
\usepackage{graphicx}
\usetheme{Frankfurt}
\usecolortheme{beaver}

\title{Entity-based keyword search for web documents}
\author{Enrico Sartori}
\date{Academic Year 2010 - 2011}
\institute{University of Trento}

\begin{document}

\begin{frame}
\titlepage
\end{frame}

\section{Introduction}
\subsection{intro}

\begin{frame}
\frametitle{Motivation}
Most of current search engines follow a keyword-based approach to
document representation\\
{\color{red}\bfseries{Advantages:}}\\
\begin{itemize}
\item Well-known approach, provides good results for information
  retrieval and document classification
\item Easy computation of similarity measures between documents, via
  transformation into vector-space objects
\end{itemize}
{\color{red}\bfseries{Drawbacks:}}\\
\begin{itemize}
\item{{\bfseries Ambiguity in natural language:}} References to the same real world
  object are expressed using different keywords
\item{{\bfseries Loss of information:}} A flat list of words can't provide
  information about relationships existing among the objects appearing
  in the document
\end{itemize}
\end{frame}

\section{Problem}
\subsection{problem}

\begin{frame}
\frametitle{Problem definition}
Given a user query in the form of a flat list of keyword:
$$
q = \{kw_{1}, \dots, kw_{n}\}
$$
\begin{enumerate}
\item A document is considered relevant if it contains references to the
same \emph{entity} named in the query
\item Ranking of documents in the results set computed considering
  the relationships found in the documents matching the query
\end{enumerate}
\begin{exampleblock}{Entity}
An \emph{entity} is every real world object which can be uniquely
identified, like a city, an enterprise or a public person.
\end{exampleblock}
\end{frame}

\section{Solution}
\subsection{Document Representation}

\begin{frame}
\frametitle{Entities and relationships}
Entity-aware document model:
\begin{itemize}
\item Resolution of groups of keyword into \emph{uniquely identified
  entities} (Named Entity Recognition)
\item Enriching original document text with the identifiers given to the entities
\end{itemize}
Relationships identification:
\begin{itemize}
\item Each sentence in the document is summarized into a set of
  triples
$$
t = (subject, verb, object)
$$
\item The verbs in the sentence describe actions that the objects in
  the document are concurring to perform.
\item The \emph{subject} and \emph{object} fields in the query can
  either be an entity Id or common keywords.
\end{itemize}
The document is represented as the set of triples produced by the
analysis of all the sentences
\end{frame}

\subsection{Query Processing}

\begin{frame}
\frametitle{From a keyword query to a set of triples}
The user query needs to be translated into a form comparable to the
document representation
\begin{itemize}
\item Entity recognition
\item Relationship identification
\end{itemize}
\vspace{5mm}
{\color{red}\bfseries{Algorithm overview:}}\\
\vspace{3mm}
\begin{tabular}{lcc}
User query as a set of keyword && $kw_{1}, kw_{2}, kw_{3}, kw_{4}$\\
& $\Downarrow$ &\\
Verbs and nouns recognition && $n_{1}, v_{1}, n_{2}, n_{3}$\\
& $\Downarrow$ &\\
Triples production && $(n_{1}, v_{1}, n_{2}), (n_{1}, v_{1}, n_{3})$\\
& $\Downarrow$ &\\
Entities recognition && $(e_{1}, v_{1}, n_{2}), (e_{1}, v_{1}, e_{2})$\\
\end{tabular}
\end{frame}

\subsection{Query Answering}

\begin{frame}
\frametitle{Query Answering}
Similarity is computed considering how deeply the triples of the
document match the query
\begin{columns}
\column{.5\textwidth}
lolol
\column{.5\textwidth}
\begin{itemize}
\item Set A: totally matching triples
\item Set B: triples matching with \emph{subject} and \emph{object}
\item Set C: triples matching only with \emph{subject} or \emph{object}
\end{itemize}
\end{columns}
{\color{red}\bfseries{Similarity measure:}}
$$
s_{1} = \frac{|\mathcal{A}|}{|\mathcal{D}|};\;
s_{2} = \frac{|\mathcal{B}|}{|\mathcal{D}| - |\mathcal{A}|};\;
s_{3} = \frac{|\mathcal{C}|}{|\mathcal{D}|-(|\mathcal{A}|+|\mathcal{B}|)}
$$

$$
s (t, d) = s_{1} + (1 - s_{1}) \times
[s_{2} + (1 - s_{2}) \times s_{3}]
$$
\end{frame}

\subsection{Indexing}

\begin{frame}
\frametitle{Fast access to data}
Compute the cardinality of the subset A, B and C can be a bottleneck
in term of performance
\begin{itemize}
\item Indexing structure living in main memory
\item Provides support for fast filtering of triples
\end{itemize}
\vspace{5mm}
The index should implement the following functions:\\
\vspace{2mm}
\begin{center}
\begin{tabular}{ll}
Set A: & $i_{\mathcal{A}} (s,v,o,d) = cnt_{s,v,o,d}$\\
Set B: & $i_{\mathcal{B}} (s,o,d) = cnt_{s,o,d}$\\
Set C: & $i_{\mathcal{C}} (s,o,d) = cnt_{s,d} + cnt_{o,d}$\\
\end{tabular}
\end{center}
\vspace{3mm}
Where $s$ is value for \emph{subject}, $o$ for \emph{object}, $v$ is
the \emph{verb} and $d$ is the Id of the current document
\end{frame}

\section{Implementation}
\subsection{Entity recognition}

\begin{frame}
\frametitle{OpenCalais and Entity Recognition}
{\color{red}\bfseries{Document Pre-processing}}\\
Need for a service that can highlight \emph{entities} into a document
and provide identifiers for them
\begin{itemize}
\item Many services available on the Internet performing this task
\item Integration of the OpenCalais REST webservice
\item Grants high quality results and constant updates to its data
\item Calling an external service presents latency issues
\end{itemize}
\vspace{2mm}
{\color{red}\bfseries{Query Answering}}\\
When responding a query time is a precious resource, calling an
external service introduces too much latency
\begin{itemize}
\item Entity resolution using a local cache of OpenCalais data
\item Cache built at document pre-processing time, storing information
  about the entities recognized by the external service
\end{itemize}
\end{frame}

\subsection{Relationships Identification}

\begin{frame}
\frametitle{Query answering}
\end{frame}

\subsection{Indexing}

\begin{frame}
\frametitle{Hexastore adaptation}
\end{frame}

\section{Experiments}
\subsection{Document preprocessing}

\begin{frame}
\frametitle{Document preprocessing time}
\end{frame}

\subsection{Query answering}

\begin{frame}
\frametitle{Query processing and answering}
\end{frame}

\subsection{Indexing}

\begin{frame}
\frametitle{Indexing time and memory}
\end{frame}

\section{Conclusions}
\subsection{Conclusions}

\begin{frame}
\frametitle{Conclusions and future work}
\end{frame}

\section*{}
\begin{frame}
\begin{center}
{\bfseries The End\\}
Thank you for your attention!\\
Questions?
\end{center}
\end{frame}

\end{document}
